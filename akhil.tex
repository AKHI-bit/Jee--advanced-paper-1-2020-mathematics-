\documentclass{article}
\usepackage{gvv}

\begin{document}

\section{\underline{\textbf{MATHEMATICS}}}\\
\date{}\\

Q1. For a Comple number z, let Re(z) denote the real part of z. Let S be the set of all complwx numbers z satisfying $z^4 - |z|^4 = 4iz^2$, where $i = \sqrt{-1}$.Then the minimum possible value of $|z_1 - z_2|^2$, where $z_1,z_2 \belongs S$ with Re($z_1$)>0 and Re($z_2$)<0, is \underline{\hspace{2cm}}.\\
\medskip

Q2.The probability that a missile hits a target successfully is 0.75. In order to destroy the target 
completely, at least three successful hits are required. Then the minimum number of missiles that
have to be fired so that the probability of completely destroying the target is $\textbf{NOT}$ less than 0.95, is \underline{\hspace{2cm}}.
\medskip

Q3. Let 𝑂 be the centre of the circle $x^2 + y^2 = r^2$,where $r>\frac{\sqrt{5}}{2}$
. Suppose 𝑃𝑄 is a chord of this circle
and the equation of the line passing through 𝑃 and 𝑄 is $2x + 4y = 5$. If the centre of the 
circumcircle of the triangle 𝑂𝑃𝑄 lies on the lies on the line $x + 2y = 4$, then the value of 𝑟 is \underline{\hspace{2cm}}\\.
\medskip

Q4.The trace of a square matrix is defined to be the sum of its diagonal entries. If 𝐴 is a 2 × 2 matrix 
such that the trace of 𝐴 is 3 and the trace of $A^3$
is −18, then the value of the determinant of 𝐴 is 
\underline{\hspace{2cm}}.\\
\medskio

Q5. Let the functions f : (-1,1) \to R and g : (-1,1) \to (-1,1) be defined by\\
       \smallskip
       $f(x) = \mid2x - 1\mid + \mid2x + 1\mid$ and $g(x) = x - [x]$,\\
       \smallskip

where $\left[x]\right$ denotes the greatest integer less than or equal to 𝑥. Let $f ^\circ g: (-1,1) \to ℝ$ be the 
composite function defined by $(f ^\circ g)(𝑥) = f(g(x))$. Suppose 𝑐 is the number of points in the 
interval (-1,1) at which $f ^\circ g$ is $\textbf{NOT}$ continuous, and suppose 𝑑 is the number of points in the 
interval (-1,1) at which $f ^\circ g$ is$\textbf{NOT}$ differentiable. Then the value of$ 𝑐 + 𝑑$ is \underline{\hspace{2cm}}.
\medskip

Q6. The value of the limit
         \smallskip
	 $lim_{x \to \frac{\pi}{2}}\frac{4\sqrt{2}(sin3x + sinx)}{(2 sin2xsin\frac{3x}{2} + cos\frac{5x}{2}) - (\sqrt{2}+\sqrt{2}cos2x + cos\frac{3x}{2}}$ is \underline{\hspace{1cm}}.\\
	 \medskip

Q7. Let 𝑏 be a nonzero real number. Suppose f: ℝ $\to$ ℝ is a differentiable function such that f(0) = 1.
If the derivative f' of f satisfies the equation
               \smallskip
	       $f'(x) = \frac{f(x)}{b^2 + x^2}$
	       \smallskip
for all x $\in$ R, then which of the following statements is/are TRUE?

(A) If 𝑏 > 0, then f is an increasing function\\
(B) If 𝑏 < 0, then f is a decreasing function \\
(C) $f(x)f(-x) = 1$ for all x $\in$ R\\
(D) $f(x) - f(-x) = 0$ for all x $\in$ R
\medskip

Q8. Let 𝑎 and 𝑏 be positive real numbers such that 𝑎 > 1 and 𝑏 < 𝑎. Let 𝑃 be a point in the first 
quadrant that lies on the hyperbola $\frac{x^2}{a^2} - \frac{y^2}{b^2}$. Suppose the tangent to the hyperbola at 𝑃 passes 
through the point (1, 0), and suppose the normal to the hyperbola at 𝑃 cuts off equal intercepts on 
the coordinate axes. Let $\triangle$ denote the area of the triangle formed by the tangent at 𝑃, the normal at 𝑃
and the x-axis. If e denotes the eccentricity of the hyperbola, then which of the following statements 
is/are TRUE?

(A)$1< e <\sqrt{2}$ \quad (B)$\sqrt{2}< e <2$ \quad (C) $\triangle = a^4$ (D) $\triangle = b^4$\\
\medskip

Q9. Let f:R $\to$ R and g: R $\to$ R be functions satisfying
\smallskip
$f(x + y) = f(x) + f(y) + f(x)f(y)$ and $f(x) = xg(x)$
for all x, y $\in$ R. If $\lim_{x \to 0}g(x) = 1$
,then which of the following statements is/are TRUE?\\
\smallskip
(A) f is differentiable at every x $\in$ R\\
(B) If 𝑔(0) = 1, then g is differentiable at every x$\in$ R\\
(C) The derivative f'(1) is equal to 1\\
(D) The derivative f'(0) is equal to 1
\medskip

Q10.Let $\alpha,\beta,\gamma,\delta$ be real numbers such that $\alpha^2 + \beta^2 + \gamma^2 \neq 0$ and $\alpha + \gamma = 1$. Suppose the point (3,2,-1) is themirror image of the point (1,0,-1) with respect to theplane ${\alpha}x + {\beta}y + {\gamma}z = \delta$. 
Then which of the following statements is/are TRUE?\\
\medskip

(A)$\alpha + \beta = 2$ \qquad (B)$\delta - \gamma = 3$\\
(C)$\delta + \beta = 4 $ \qquad (D)$\alpha + \beta + \gamma = \delta$\\
\medskip

Q11.Let a and b be positive real numbers. Suppose $\overrightarrow {PQ} = a\hat{i} + b\hat{j}$ and $\overrightarrow{PQ} = a\hat{i} - b\hat{j}$ are adjacent sides 
of a parallelogram PQRS. Let $\overrightarrow{u}$ and $\overrightarrow{v}$ be the projection vectors of $\overrightarrow{w} = \hat{i} + \hat{j}$ along $\overrightarrow{PQ}$ and $\overrightarrow{PS}$ respectively. If $|\vec{\overrightarrow{u}}| + |\vec{\overrightarrow{v}}| = |\vec{\overrightarrow{w}}|$ and if the area ofthe parallelogram PQRS is 8, then which of the following statements is/are TRUE?\\
\medskip
(A) $a + b = 4$\\
(B) $a - b = 2$\\
(C) The length of the diagonal PR of the parallelogram PQRS is 4\\
(D) 𝑤⃗ is an angle bisector of the vectors $\overrightarrow{PS}$\\ and $\overrightarrow{PS}$\\
\medskip

Q12.For nonnegative integers s and r, let 
          
	     $\binom{s}{r} =
\begin{cases}
\frac{s!}{r!(s - r)!}, & \text{if } r \leq s, \\
0, & \text{if } r > s.
\end{cases}$\\
\medskip
For positive integers m and n, let g
          
	  $g(m, n) = \sum_{p=0}^{m+n} f(m, n, p) \binom{n + p}{p}$\\
	  \medskip
where for any nonnegative integer p,

       $ f(m, n, p) = \sum_{i=0}^{p} \binom{m}{i} \binom{n + i}{p} \binom{p + n}{p - i}$\\
       \medskip
Then which of the following statements is/are TRUE?
\medskip

(A) $\( g(m, n) = g(n, m) \)$ for all positive integers $\( m, n \)$\\  

(B) $\( g(m, n + 1) = g(m + 1, n) \)$ for all positive integers $\( m, n \)$\\  

(C) $\( g(2m, 2n) = 2 g(m, n) \)$ for all positive integers $\( m, n \)$\\  

(D) $\( g(2m, 2n) = (g(m, n))^2 \)$ for all positive integers $\( m, n \)$\\
	  \medskip
Q13.An engineer is required to visit a factory for exactly four days during the first 15 days of every
month and it is mandatory that \textbf{no} two visits take place on consecutive days. Then the number of all 
possible ways in which such visits to the factory can be made by the engineer during 1-15 June 
2021 is $\underline{\hspace{2cm}}.$
\medskip

Q14.In a hotel, four rooms are available. Six persons are to be accommodated in these four rooms in 
such a way that each of these rooms contains at least one person and at most two persons. Then the
number of all possible ways in which this can be done is $\underline{\hspace{2cm}}.$
\medskip

Q15.Two fair dice, each with faces numbered 1, 2, 3, 4, 5 and 6, are rolled together and the sum of the 
numbers on the faces is observed. This process is repeated till the sum is either a prime number or a perfect square. Suppose the sum turns out to be a perfect square before it turns out to be a prime number. If  is the probability that this perfect square is an odd number, then the value of 14p is $\underline{\hspace{2cm}}.$
\medskip

Q16.Let the function $f:\left[0, \right1] \to R$ be defined by

      $f(x) = \frac{4^x}{4^x + 2}$\\
      \smallskip
Then the value of\\
      \smallskip
        $f\left(\frac{1}{40}\right) + f\left(\frac{2}{40}\right) + f\left(\frac{3}{40}\right) + \dots + f\left(\frac{39}{40}\right) - f\left(\frac{1}{2}\right)$ 
	is \underline{\hspace{2cm}}.\\
\medskip
Q17. Let f: R $\to$ R be a differentiable function such that its derivative f'is continuous and f(\pi) = −6 
If $F:[0, \pi]$ $\to$ R is defined by $F(x) = \int_{0}^{x}f(t) \, dt$
    \smallskip
    $\int \left( f'(x) + F(x) \right) \cos x \, dx = 2
    $
then the value of 𝑓(0) is $\underline{\hspace{2cm}}$.
\medskip

Q18. Let the function $f: (0, \pi) \to R$ be defined by\\
      $f(\theta) = (sin\theta + cos\theta)^2 + (sin\theta - cos\theta)^4.$\\
      Suppose the function f has a local minimum at $\theta$ precisely when $\theta$ $\in$\\
       
       ${ \lambda_1 \pi, \dots, \lambda_r \pi}$, where  $0<\lambda_1 < \cdots < \lambda_r < 1$. Then the value of $\lambda_1 + \dots + \lambda_r$ is $\underline{\hspace{1cm}}.$
        
     
\end{document}
